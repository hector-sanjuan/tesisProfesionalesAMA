\documentclass{article}
\usepackage{graphicx} % Required for inserting images

\title{librodeTesisProf}
\author{Hector SanjuanG}
\date{February 2026}

\usepackage[paperwidth=13.5cm,paperheight=21cm,margin=1.3cm]{geometry}
\usepackage{fontspec}
\setmainfont{QTBasker}

\begin{document}

20 ÁNGELES MENDIETA ALATORRE
 
Las facultades que impartían cátedra eran: las de Artes,
Teologia, Canones, Leyes, y a partir de 1590 también la de
Medicina. Cada una exigía  un signo distintivo en la escla-
vina o en la muceta de sus doctores: azul, blanca, verde,
roja y amarilla, respectivamente.

Mandaban los estatutos para la tesis fuera impresa, para 
fijarla en las puertas de la Universidad. El examen de ba-
chiller, pese a la severidad ceremonial, era muy sencillo en 
comparación con las fiestas de graduación superior. El ba-
chiller debía estar de pie, descubierto y junto a él, los dos 
bedeles; en esta postura solicitaba el grado. Entonces el doc-
tor se lo concedía mediante una fórmula latina. 

Después de tres o cuatro anos, el bachiller podía solicitar 
la licenciatura. Entonces se sometía a dos exámenes, uno de 
ellos era público, llamado "repetición". Por cierto, afirma 
el doctor de la Maza, "con notable exactitud semántica, pues 
no consistía sino en repetir de memoria, con mayor o menor 
agudeza de ingenio, los datos aprendidos con anterioridad".
En este examen, el licenciado proponía sus conclusiones, ha-
ciéndose el acto en el general que era el salón de ceremonias.
Duraba el examen una hora y después, un niño, no mayor 
de doce años, abría con un cuchillo varias páginas de ciertos 
libros clásicos de la especialidad, para que fueran repetidos 
textualmente por el estudiante. 

El arca de la Universidad recogía las propinas que pre-
viamente repartía el licenciado y cuyo costo era de 600 pe-
sos, cantidad fabulosa para aquel entonces. Luego se abría 
una urna en la que los examinadores habían echado las A 
y las R de plata que aprobaban o reprobaban. Venía al día 
siguiente la entrega del título, conforme lo "explican sabro-
samente los viejos estatutos palafoxianos". A las diez se pre-
sentaba el maestro de ceremonia, el secretario, los doctores 
y partía el séquito con trompetas y a caballo para que el 
licenciado pidiera el grado en la Iglesia Catedral. Allí per-
manecía también descubierto y del lado izquierdo, para re-
gresar, una vez concedido éste, al lado derecho del decano, a
continuar el ceremonial  que terminaba con una fiesta en 
casa del rector.

\maketitle

\section{Introduction}

\end{document}
